\documentclass[margin,line]{res}

\usepackage{apacite}
\usepackage{bibentry}
\usepackage{enumitem}

\oddsidemargin -.5in
\evensidemargin -.5in
\textwidth=6.0in
\itemsep=0in
\parsep=0in
% if using pdflatex:
%\setlength{\pdfpagewidth}{\paperwidth}
%\setlength{\pdfpageheight}{\paperheight}

\newenvironment{list1}{
  \begin{list}{\ding{113}}{%
      \setlength{\itemsep}{0in}
      \setlength{\parsep}{0in} \setlength{\parskip}{0in}
      \setlength{\topsep}{0in} \setlength{\partopsep}{0in}
      \setlength{\leftmargin}{0.17in}}}{\end{list}}
\newenvironment{list2}{
  \begin{list}{$\bullet$}{%
      \setlength{\itemsep}{0in}
      \setlength{\parsep}{0in} \setlength{\parskip}{0in}
      \setlength{\topsep}{0in} \setlength{\partopsep}{0in}
      \setlength{\leftmargin}{0.2in}}}{\end{list}}

\pagenumbering{gobble}

\begin{document}

\nobibliography{larsen-cv.bib}
\bibliographystyle{abbrv}

\name{Matthew C. Larsen \vspace*{.1in}}

\begin{resume}
\section{\sc Contact Information}
\vspace{.05in}
%\begin{tabular}{@{}p{2in}p{4in}}
%7000 East Ave     & {\it Voice:}  (530) 902-1033 \\
%Mail Stop L-38   &  {\it E-mail:}  larsen30@llnl.gov\\
%Livermore, CA 94550         &  {\it WWW:} www.mclarsen.com\\
%                                 &  \\
%\end{tabular}

\begin{tabular}{@{}p{2in}p{4in}}
2617 College Ave     & {\it Voice:}  (530) 902-1033 \\
Livermore, CA 94550   &  {\it E-mail:}  larsen.matt1@gmail.com\\
                      &  {\it WWW:} www.mclarsen.com\\
                                 &  \\
\end{tabular}

%\section{\sc Objective}
%To continue my professional career as a researcher and computer scientist.
\section{\sc Research Interests}
Scientific visualization, in-situ analysis infrastructures, high performance computing, ray tracing, volume rendering, computer graphics, performance modeling, GPU and many-core programming

\section{\sc Education}
\textbf{Ph.D Computer Science} - University of Oregon - Advisor: Hank Childs (2016)
\\
\textbf{M.S. Computer Science} - University of Oregon (2015)
\\
\textbf{B.S. Computer Science} - CSU Sacramento - Magna Cum Laude (2012)%- GPA 3.823 %- December 2012

\section{\sc Work Experience}

\begin{tabular}{lll}
\textbf{Staff Scientist}	& Lawrence Livermore National Laboratory & 1/16-Pres \\
\textbf{Intern Researcher}	& Lawrence Livermore National Laboratory & 6/15-9/15 \\
\textbf{Intern Researcher}	& Oak Ridge National Laboratory & 6/14-9/14 \\
\textbf{Graduate Research Fellow}	& University of Oregon  & 9/13-12/15 \\
\textbf{Developer}	& POS Portal, Sacramento, Ca.  & 2/13-9/13 \\
\end{tabular}


\section{\sc Awards}
\begin{itemize}
	\item Best Paper Award: “Interactive In Situ Visualization and Analysis using Ascent and Jupyter”
  At the Workshop on In Situ Enabling Extreme-Scale Analysis and Visualization (ISAV) 2019.
	\item Best Paper Award: “A Flexible System for In Situ Triggers.” At the Workshop on In Situ
	Infrastructures for Enabling Extreme-Scale Analysis and Visualization (ISAV) 2018.
	\item Best Paper Finalist: “Performance Modeling of In Situ Rendering.” At the International Conference
	for High Performance Computing, Networking, Storage and Analysis (SC) 2016.
	\item J. Donald Hubbard Family Scholarship in Computer and Information Science, University of Oregon (2015)
\end{itemize}


%\section{\sc Peer-Reviewed Journal Articles}
%\begin{enumerate}
%
%	\item \bibentry{testbib}.
%	\item \bibentry{DBLP:journals/cga/MorelandSULMPKS16}.
%	\item \bibentry{Moreland:2015:VEP:3026759.3026765}.
%\end{enumerate}
%\section{\sc Peer-Reviewed Conference and Symposiums}
%\begin{enumerate}[resume]
%	\item \bibentry{Marsaglia:EGPGV19}.
%	\item \bibentry{Kress:ISC19}.
%	\item \bibentry{Labasan:IPDPS19}.
%	\item \bibentry{8231845}.
%	\item \bibentry{pgv.20171088}.
%	\item \bibentry{7877102} (\textbf{Best Paper Finalist}).
%	\item \bibentry{7874308}.
%	\item \bibentry{7348074}.
%	\item \bibentry{pgv.20151155}.
%	\item \bibentry{7156388}.
%\end{enumerate}
%
%\section{\sc Peer-Reviewed Workshops}
%\begin{enumerate}[resume]
%	\item \bibentry{Larsen:2018:FSS:3281464.3281468} (\textbf{Best Paper}).
%	\item \bibentry{Larsen:2017:ASI:3144769.3144778}.
%	\item \bibentry{10.1007/978-3-030-17872-7_12}.
%	\item \bibentry{Li:2017:PIS:3144769.3144773}.
%	\item \bibentry{Larsen:2015:SBS:2828612.2828625}.
%\end{enumerate}
\section{\sc Peer-Reviewed First/Last Authored Publications}
\begin{enumerate}
	\item \bibentry{jits_complicated}.
	\item \bibentry{Larsen:2018:FSS:3281464.3281468} (\textbf{Best Paper}).
	\item \bibentry{Larsen:2017:ASI:3144769.3144778}.
	\item \bibentry{7877102} (\textbf{Best Paper Finalist}).
	\item \bibentry{7874308}.

	\item \bibentry{Larsen:2015:SBS:2828612.2828625}.
	\item \bibentry{pgv.20151155}.
	\item \bibentry{7156388}.


\end{enumerate}
\section{\sc Other Peer-Reviewed Publications}
\begin{enumerate}[resume]
	\item \bibentry{10.1145/3426462.3426469}.
	\item \bibentry{doi:10.1177/1094342020935991}.
	\item \bibentry{9308046}.
	\item \bibentry{darkroom}.
	\item \bibentry{isckress20}.
	\item \bibentry{jupyter} (\textbf{Best Paper}).
	\item \bibentry{testbib}.
	\item \bibentry{Kress:ISC19}.
	\item \bibentry{Labasan:IPDPS19}.
	\item \bibentry{Marsaglia:EGPGV19}.
	\item \bibentry{10.1007/978-3-030-17872-7_12}.
	\item \bibentry{8231845}.
	\item \bibentry{Li:2017:PIS:3144769.3144773}.
	\item \bibentry{pgv.20171088}.
	\item \bibentry{DBLP:journals/cga/MorelandSULMPKS16}.
	\item \bibentry{7348074}.
	\item \bibentry{Moreland:2015:VEP:3026759.3026765}.
\end{enumerate}

\section{\sc Software Artifacts}
{\em \textbf{Ascent (primary developer since inception in 2017}) }
\begin{itemize}
	\item Ascent is a flyweight library for in situ visualization and analysis, with emphases on minimal
	memory usage and API, as well as interoperability with external libraries.
	\item Larsen's contributions include design on the main runtime, which translates the front-facing user API into a data flow network, the expressions language, and in situ triggers infrastructure. Ascent has been integrated into numerous simulation codes, and Ascent has been demonstrated running using over 16K GPUs on LLNL's Sierra supercomuter.
\end{itemize}

{\em \textbf{Devil Ray(software architect and developer since inception in 2017}) }
\begin{itemize}
	\item Devil Ray is visualization and analysis libary for high-order finite elements targeting modern HPC
  architectures. Originally, Devil Ray was developed as high-order element ray tracer, but Devil Ray has grown to include general analysis and visualization capabilites responding to the needs of our customers.
	\item As a staff scientist interacting with LLNL's simulation codes, Larsen identified gaps in our
  current visualization and analysis capabilities. Larsen secured internal funding to develop Devil Ray
  and implemented the library along with Masado Ishii, a student funded by LLNL. To date, we have demonstrated
  Devil Ray running concurrenlty on over 4,000 GPUs is support of LLNL's mission.
\end{itemize}

{\em \textbf{Stawman In Situ Visualization Mini-app (primary developer 2016-2017)} }

\begin{itemize}
	\item Strawman ia a prototype in situ visualization infrastructure mini-app. The main projects goals were ease of use (i.e., low code footprint for integrations) and execution in distributed-memory parallel batch environments. Strawman is the precursor to Ascent.
	\item As an intern at LLNL, Larsen's primary contributions included integrating technologies Conduit, EAVL, and the IceT parallel image compositing library into a infrastructure capable of leveraging many-core architectures on leading supercomputers. Responsibilities included visualization and parallel rendering implementation. To demonstrate its capabilities, Strawman was integrated with three physics mini-apps (LULESH, Kripke, and Cloverleaf3D), and performed weak scaling studies up to 4096 cores and 128 GPUs.
\end{itemize}

{\em \textbf{VTK-m (participant since inception in  2016)} }
\begin{itemize}
	\item VTK-m is a library for portably performance over many-core architectures. It is a follow-on effort
	to the popular Visualization ToolKit (VTK), with the “m” indicating many-core support.
	\item Larsen is the primary contributor of the rendering component of VTK-m, developing a portable ray-tracing framework that is capable of structured and unstructured volume rendering, radiography, and surface rendering. Additionally, Larsen has contributed various core components of VTK-m such as support for atomic operations.
\end{itemize}

{\em \textbf{VTK-h (primary developer since inception in  2016)} }
\begin{itemize}
	\item VTK-h is a library built on top of VTK-m that includes a distributed memory component, with the “h” indicating hybrid-parallel support. The VTK-h library has served as a production prototype for distributed-memory support inside of VTK-m.
	\item Larsen has developed the composable distributed-memory filter system built using components from VTK-m. Other contributions include a hybrid-parallel image compositing system for surface and volume rendering leveraging DIY2.
\end{itemize}

{\em \textbf{ROVER (primary developer since inception in 2016)} }
\begin{itemize}
	\item ROVER is set of distributed-memory components that performs multi-group simulated radiography on simulation meshes.
	\item Larsen's contributions include a generalized multi-group absorption and emission distributed-memory compositing infrastructure. Various components of ROVER have been integrated into Ascent and VTK-m. Additionally,
	Larsen is the architect of a high-order element ray tracing component, and he has been supervising and advising student contributions.
\end{itemize}

{\em \textbf{VisIt (participant since 2016)} }
\begin{itemize}
	\item VisIt is developed by over a dozen developers, is used at supercomputing centers around the
	world, has been downloaded more than 200,000 times, and was recognized with an R\&D 100
	award in 2005.
	\item Larsen's contributions include adding enhancements to the pick infrastructure, rendering, VTK-m integration, and various database readers.
\end{itemize}

{\em \textbf{EAVL (participant from 2014-2016)} }
\begin{itemize}
	\item EAVL (The Extreme-Scale Analysis and Visualization Library) is a portable performance visualization that served as one of the three prototypes for VTK-m.
	\item As student, Larsen implemented ray-tracing, volume rendering, and radix sort algorithms (OpenMP and CUDA) to the EAVL library. Additionally, Larsen published various research papers showing that algorithms based on data-parallel primitives can be competitive with algorithms that target specific HPC architectures.
\end{itemize}





\section{\sc Peer Reviewed conference tutorials}
\begin{list2}
  \item ``In Situ Scientific Analysis and Visualization using ALPINE Ascent'' at an ECP Training Event - Dec 2020, Virtual
	\item ``In Situ Analysis and Visualization with SENSEI and Ascent'' at ACM/IEEE SuperComputing 2020, Atlanta, GA, November, 2020.
	\item ``In Situ Visualization and Analysis with Ascent'' at Exascale Computing Project (ECP) Annual
	Meeting, Houston, TX, February 2020.
	\item ``In Situ Analysis and Visualization with SENSEI and Ascent'' at ACM/IEEE SuperComputing 2019, Denver, CO, November, 2019.
	\item ``In Situ Visualization and Analysis with Ascent'' at Exascale Computing Project (ECP) Annual
	Meeting, Houston, TX, January 2019.
\end{list2}

\section{\sc Keynote Presentations}
\begin{list2}
	\item ISC Workshop on In Situ Visualization (WOIV), ``The changing balance in HPC and in situ visualization challenges,''  Frankfurt, Germany, June 2018.
\end{list2}
\section{\sc Invited Talks}
\begin{list2}
	\item Oak Ridge National Laboratory, ``Devil Ray: A Portably Performant Ray Tracer for High-Order Element Visualization'' Oak Ridge, TN, October 2019.
	\item University of Oregon, ``In Situ Visualization for Exascale Computing,'' Eugene, OR, November 2018.
	\item Rheinisch-Westfaelische Technische Hochschule (RWTH)-Aachen, ``The changing balance in HPC and in situ visualization challenges,'' Aachen, Germany, July 2018.
	\item Los Alamos National Laboratory, “Performance Modeling of In Situ Rendering,” Los Alamos, NM, September 2016.
	\item Texas Advanced Computing Center, ``Ray Tracing Within a Data Parallel Framework,'' Austin, TX. January 2015.
	\item  Kitware, Inc., ``Ray Tracing Within a Data Parallel Framework,'' Clifton Park, NY. December 2014.
\end{list2}
\section{\sc Invited Seminars}
\begin{list2}
	\item Schloss Dagstuhl Seminar on Scientific Visualization,  Wadern,
	Germany, July 2018.
\end{list2}

\section{\sc Professional Service}
\begin{list2}
	\item Organization
		\begin{list2}
			\item Program Co-Chair: ISAV Nov 14 June 2021
			\item Program Co-Chair: EGPGV co-located with EuroVis, Zurich, Switzerland, 14 June 2021
      \item Khronos ANARI adivory panel 2021
			\item Co-organizer: Visualization in Practice, IEEE Vis associated event, New Orleans, LA, USA
 October 2021
			\item Co-organizer: Visualization in Practice, IEEE Vis associated event, Salt Lake City, USA, October 2020
			\item Co-organizer: Visualization in Practice, IEEE Vis associated event, Vancouver, Canada, October 2019
			\item Site chair: DOE Computer Graphics Forum, Monterey, CA, April 2019
			\item Site chair: VTK-m hackathon, Livermore, CA, August 2018
		\end{list2}
	\item Program Committee
		\begin{list2}
			\item ACM/IEEE Supercomputing (SC) tutorials 2017, 2018, 2019
			\item Eurographics Parallel Graphics and Visualization Symposium 2018, 2019, 2020
			\item In Situ Infrastructures for Enabling Extreme-scale Analysis and Visualization (ISAV) 2017, 2018, 2019, 2020
% CAES Data science boot camp presentation on VisIt + Ascent 2020, 2021
		    \item Visualization in Practice 2017
		\end{list2}
	\item Technical Paper Reviews
	\begin{list2}
		\item IEEE Transactions on Visualization and Computer Graphics(TVCG) 2020
		\item Journal of Computational Science 2018
		\item Journal of Computer Graphics and Applications 2017
		\item EuroVis 2017 2021
		\item Pacific Vis 2017
		\item SIAM Journal on Scientific Computing 2016
		\item EuroGraphics Parallel Graphics and Visualization Symposium 2015, 2018, 2019, 2020
		\item IEEE Visualization Conference 2015, 2019, 2021
		\item In Situ Infrastructures for Enabling Extreme-scale Analysis and Visualization  2015, 2017, 2018, 2019, 20202
	\end{list2}
	\item Grant Reviews
	\begin{list2}
    \item Panelist for DOE Office of Advanced Scientific Computing Research funding opportunity, 2019
	\end{list2}
	\item Other Service
	\begin{list2}
		\item ACM/IEEE SuperComputing Visualization Showcase 2018, 2019 (Judge)
	\end{list2}
\end{list2}
\end{resume}


\end{document}




